\documentclass{sig-alternate}
\begin{document}
\title{Visual Formalism, Liturature Review}
\numberofauthors{1}
\author{
\alignauthor
Stevenson Gossage\\
       \affaddr{School of Computer Science, Carleton University}\\
       \affaddr{1125 Colonel By Drive, Ottawa, ON K1S 5B6}\\
       \affaddr{Ottawa, Canada}\\
       \email{sgossage@connect.carleton.ca}
}
\date{21 April 2011}

\maketitle

\terms{Visual Formalism, Distributed Collaboration}

\keywords{Visual Formalism, Distributed Collaboration}

\section{Introduction}
\subsection{Motivation}
\section{Summary of the reviewed Papers}
\subsection{Sumary: Information Visualization\cite{Gershon:1998:Informationvisualization}}
This paper explains that a visualization is a technique used to display
information in a graphical or visual manner with the intent of leveraging our 
natural visual capabilities. Visualization allows us to process, analyse and
understands massive amounst of data very quickly. The problem with
visualizations is that we are trying to use them, to quickly understand more and
more abstract data, that does not always have an immediately obvious mapping to
the real world. This presents a challange in trying to develop reusable
meaningful visualisations. There is ongoing reasearch to try and leverage
metaphors and to try understand what type of analytic tasks could be enhance by
visualizations based on those metaphors.

The paper talks about visualization developments in several areas including,
document and software visualiztion, distortion techniques, hierarchies, data
integration (including datawharehousing and data mining) and world wide web. 
At the time of publication of this paper in the late 90's there was also some
real world commercial applications including applications based on work done at
Xerox PARC including Perspective Wall, Cone Tree and Wide Widgets.

The author suggests that, at the time, there was still much work to be done
in terms of understanding the potential and usefullness of the technology but it
was clear that visualizations would play a huge role in the future of computing.
Gershon describes visualization as an immerging discipline which must progress
through four fundamental stages and suggests that visualization is progressing
at an accelerated pace because many of the step are are being progressing in
parallel; ``An emerging discipline progresses through four stages. First, it
starts as a craft, practiced by artisans using heuristics. Second, researchers
formulate scientific principles and theories to gain insights about the
processes. Third, engineers eventually refine these principles and insights
into production rules. Fourth, the technology becomes widely available.''

The paper describes that widespread usefulness of visualizations will be
directly influenced by the content providers' (software design and developers)
ability to create visualizations that are accessible and easily understood by
a wide range of end users, making them equaliy useful for users of diverse
backgrounds. In order to create these meaningful visualizations in real world
applications the systems being modeled and the processes envolved must be
understood. This understanding in itself could lead to the correct visualization
for any particular task. 

There ate many areas that we need to learn more about in order to better
support human learning throuch visulizations like what types of interactions
make sense for any particular visualization, how we perceive and
understand information both visually and nonvsually as well as how we search
for information and how we adapt those searches based on previous knowledge. A
better understanding of these issues and how they relate to specific analylict 
tasks will inform the creation or more usefull, meaningfull visualisations that 
that can truly help end users process more data more acuratelly and quicker. The
ability to create more flexible task oriented user interfaces can also aid in
the development more useful visualizations.

An important point that Gershon makes in the paper is that we also need to
understand when graphic visualization may infact be less appropriate. In fact
sometimes a visualization may be harder to understand than words; we need to
weigh the options and choose the representation that is most appropriate to
allow the users to accomplish the required task.

In conclusion the paper re-iterates that there is great need for the evolution
of this emerging technology especially in ``The development of scientific and
engineering principles for the generation of visualizations  (to users with
diverse needs and capabilities) and a methodology for solving problems with
information visualization are badly needed.''

\subsubsection{Key Contributions}
\subsection{Sumary: Situational Awareness Support to Enhance Teamwork
in Collaborative Environments \cite{Kulyk:2008:SituationalAwareness}}
This paper was motivated by that fact that today, many modern teams are
collaborating using a multitude of visualizations on multiple large format
displays. Durring these activities the situational awareness of the teams is not
always ideal. This paper attempts to develop some guidlines to help design
better co-located collaborative systems for use with multiple large format 
displays whith affordances for situational awareness. 

There were several concepts developed to help support situational awareness on
large shared displays including an interface they called ``Memory Board is
an interface that automatically stores and visualizes the activity history''. It
allows participants access historical annotations made on other views or
visualiztions and allows them to be aware of which member is controlling a
particular visulisation or display. They also developed the \emph{Highlighting
on Demand} interface which allows individulas to highlight or fade out any
part of any display using a personal interaction device.

The paper argues that simply providing complex visualizations across or using
multiple displays may infact increase the cognitive load and have a negative
impact on the collaborative activity. Designing situational awarness into thes
applications is necessary to properly coordinate team descision making and truly
support the collaborative process. One reason is that the participants can be
easily distracted by subtasks that should in fact be transparent to the activity but in
reality are not, like who is in control of a shared artifact or which display
or visualisation is the one currently of interest. Shared awareness of these
types of interactions are essential and can ''leads to informal social
interactions and development of shared working cultures which are essential
aspects of group cohesion''.

Kulyk states that situated awareness is concerned with the individuals
knowledge and understanding of the events, information and evironment as well as
the shared understanding of the team as a whole and their understanding of the
past and present and even its impact on future events. Kulyk's work focuses on
the following three main aspects which directly relate to and extend Endsley's
three levels of situated awareness:
\begin{itemize}
\item A person's previous knowledge and understanding which includes the
sourccde and nature of relevant events and information.
\item Detection and comprehension of the relevant perceptual cues and
information from the environment including the comprehension of multiple
visualizations and their context.
\item Interpretation of theses visulizations and the continuous reconfiguring of
understanding and knowledge during collaboration. This support the
awarness of changes in the environment as well as the ability to
keep track of work in progress.\end{itemize} 
Kulyks' is actually interested in the impact of situational awareness on team
collaboration and specifically what she refers to as \emph{shared situational
awareness}. She defines it as the level of the group or team's awareness of
individual situational awareness based the above three points. With respect to
shared situational awareness, the paper is mainly concerned with answering the
following questions:
\begin{itemize}
\item To understand the impact of situaltional awareness on
collaboration.
\item Understand how to support shared situational awarness in collaborative
envirnoments.
\item Understand how to leverage the uised of shared large format monitors to 
support chared situatinal awareness.
\item How can new interactive systems and visualisations be designed and
evaluated based on their support for situational awareness and how might these
systems acutally stimulate new and existing forms of collaboration.
\end{itemize}
The paper points outs that the level of support for situational awarness
varies depending on the collaborative activity. For example a collaborative team
that supports emergency dispatch would need a higher level of support when
compared to a team working on scientific collaboration. In the long run,
whatever the reason for the collaboration, mistakes cost and better situational
awareness support can help reduce mistakes. 
Next the paper talks specifically about situational awareness and scientific
collaboration and in particular how large shared displays could be used to 
better support situational awareness in current omics experimentation in 
molecular biology. The following key points were made:
\begin{itemize}
  \item Visualizations on a shared display encourages group discussions.
  \item The visualization of data on a shared display allows users to quickly
  dtermine the quality of the information or dataset.
  \item Multiple visualizations may be needed and when used there must be a
  clear visual relationship between these visualizations so as to avoid
  participants from getting lost, from being distrated and from change blindness
  which occurs when the participants to do not realize that the focus of the
  discussion has changed from one visualization to  another.
  \item When multiple shared displays are being used with multiple
  corelated visualizations then changes in one visualization should also trigger
  changes in the other related visualizations.
  \item Difficulties arise when team members from diverse disciplines need to 
  use the same visualizations that may have been designed to be understood by
  people with a specific background or skillset as is the case in many
  scientific visualizations.
\end{itemize}
To address some of the problems related to situational awarness for shared
displays the authors implemented Highlighting on demand interface which allowed
the person currently controlling the shared display to highlight or fade out
certain visualizations or areas of the shared display(s). This would increase
awareness amoung the team as to the relevant visulization at that moment. A
second solution was the memory board interface which stores and visualizes the 
changes in realtime during team collaboration. This ensures that the entire
collaborative process is recorded and all intermediate visualizations,
annotations etc. can be retrieved and reviewed. A control interface is
suggested as means of controlling the shared displays and visualizations and
giving access and access informaion about the visualizations and displays. This
was envisoned to run on a shared touch screen as well on the team members
individual interaction devices. Knowing who is making changes and what changes
are being made is crtical to any collaborative effort and the control interface
attempt to address this problem.
\subsubsection{Key Contributions}
\subsection{Sumary: Beyond Models and Metaphors: Visual Formalisms in User
Interface Design\cite{Nardi:BeyondModels}}
\subsubsection{Sumary}
In this paper they argue that the user interface can be designed to support
the visual representation of the underlying program's semantics and can help
offload cognitive load to facilitate problem solving. Nardi postulates that at
the time this paper was published in the early 90's it was time to start
focusing more attention on the semantic interface rather then the also
important but less interesting supply of user commands to the underlying
application. She claims that design and analysis tasks are complex and software
designed to support these activities would be enhanced with well designed
visual interface from which the semantics of the application could be easily
inferred by the users. The designers of this type of complex application would
be well served to have a toolkit of computational building blocks, such that
they could piece together the appropriate blocks which could support this
visual transfer of application semantics.

Nardi argues that neither Mental models nor Metaphors are appropriate
theoretical foundations to support such  building blocks but that these blocks
could be developed by leveraging the power of visual formalisms. She argues
that different visual formalisms can be used as the building blocks for
semantic interfaces and showed how a table or grid visual formalism was
implemented and explains the potential for it to be used as a building block.
\subsubsection{Key Contributions}
\subsection{Sumary: Heuristics for Information Visualization Evaluation}
\subsubsection{Key Contributions}
\subsection{Sumary: A Collaborative Dimensions Framework: Understanding the Mediating Role of Conceptual Visualizations in Collaborative Knowledge Work}
This paper describes how conceptual visualizations can be used as cognitive
artefacts that support collaborative knowledge work. The paper attempts to
combine the cognitive dimensions of notations framework \cite{Blackwell:NotationalSystems} ,
communicative dimensions framework 
\cite{Hundhausen:CommunicativeDimensionsFramework} and work on the boundary
object paradigm \cite{Star:BoundaryObjects} to inform the creation of a tool
used for the description of collaborative visualizations for knowledge work.
\subsubsection{Key Contributions}
\subsection{Sumary: Distributed and Collaborative Visualization}
\subsubsection{Key Contributions}
\section{Discussion}
\section{Conclusions}
\bibliographystyle{abbrv}
\bibliography{litReview}%sigproc.bib is the name of the Bibliography 
\end{document}
