% %%%%%%%%%%%%%%%%%%%%%% file typeinst.tex %%%%%%%%%%%%%%%%%%%%%%%%%%%%%%  This
% is the LaTeX source for the instructions to authors using the LaTeX document
% class SVMultln with class option 'lnbip' for contributions to the Lecture
% Notes in Business Information Processing series. www.springer.com/series/7911
% Springer Heidelberg 2007/08/05  It may be used as a template for your own
% input - copy it to a new file with a new name and use it as the basis for your
% article. It contains a few tweaked sections to demonstrate features of the
% package, though.  If you have not much experiences with Springer LaTeX
% support, you should better use the special demonstration file "lnbip.tex"
% included in the LaTeX package for LNBIP as template.
% %%%%%%%%%%%%%%%%%%%%%%%%%%%%%%%%%%%%%%%%%%%%%%%%%%%%%%%%%%%%%%%%%%%%%%%

\documentclass[lnbip,sechang,a4paper]{svmultln}
\usepackage{amssymb}
\setcounter{tocdepth}{3}    
\usepackage{graphicx}

\usepackage{url}
\urldef{\mailsa}\path|sgossage@scs.carleton.ca|
\urldef{\mailRobert}\path|robert_biddle@carleton.ca|
\usepackage[pdfpagelabels,hypertexnames=false,breaklinks=true,bookmarksopen=true,bookmarksopenlevel=2]{hyperref}
\bibliographystyle{plain}


\begin{document}

\mainmatter  

\title{CMAP: A Collaborative Multitouch Agile Planner based on Principles for Collaboration}


\titlerunning{CMAP}

\author{Stevenson Gossage and Robert Biddle}  
\authorrunning{Gossage and Biddle}
\institute{School of Computer Science, Carleton University, Ottawa, Canada\\
\mailsa, \mailRobert}

% \toctitle{TOC}
% \tocauthor{Gossage and Biddle}

\maketitle

\begin{abstract}

  Agile software development often involves two very simple but very
  effective supporting tools: the storycard and the cardwall.  This
  paper reports on our project to design a collaborative Agile planner
  based on principles for collabortion around interactive displays.
  The main goal of this project was to explore the use of multitouch
  enabled surfaces and how this technology could be leveraged to
  create a new level of usability in terms of group collaboration in
  Agile development. 

\end{abstract}

% \tableofcontents

\keywords{Agile Software Development, User Stories,  Cardwalls, Multitouch Displays}

\section{Introduction}

Agile software development often involves two very simple but very
effective supporting tools: the storycard and the cardwall. These work
so well that we need to be cautious about applying any technology to
improve them. Recent advances in interactive display technology,
however, suggest that effective collaboration might now be supported
in a more lightweight way and in harmony with Agile
practices. Moreover, excellent work has been done on identifying
principles for collaborative behaviour around tables and displays.
This paper reports on our project to design a collaborative Agile
planner based on principles for collabortion around interactive
displays.

Both story cards and cardwalls work from the basis of stories
themselves, and as Cohn says, ``The technique of expressing
requirements as user stories is one of the most broadly applicable
techniques introduced by the Agile processes'' \cite{StoriesRequ}.
Moreover, cardwalls act as what Cockburn call an {\em Information
  Radiator} \cite{InfoRad}, and help people maintain a situational
awareness of the state of the project, and also serve as place for
people to gather and discuss the project and state of the project and
quickly determine areas that need attention.  The role of physical
artefects, especially story cards and cardwalls, in Agile development
has been considered by Sharp, Robinson and Petre \cite{cite}, and they
show how there are both notational and social elements successfully at
work. Their results connect well with work in our own group, both on
social elements \cite{EWAgile} and on the nature of artefacts in
planning meetings \cite{JBAgile}. 

Large multitouch displays can be tables or walls. Their support for
handling multiple touchs allows recognition of multi-finger gestures,
such as pinching and twisting now commonplace on small-scale
multitouch devices such as the Apple iPhone.  But large-scale surfaces
also allow simultaneous use by more than one person, opening the door
to a new kind of support for collaborative applications. Principles for
the design of this kind of collaborative software have been identifies
by Scott, Grant, and Mandryk \cite{ScottGuidelines}, based on studies
of actual collaborative practices.

This paper describes the design of a software system, CMAP: A
Collaborative Multitouch Agile Planner. In this work, we began by
studying the earlier projects by Weber, Ghanam, Wang, Morgan and
Maurer at the University of Calgary \cite{Wang,Webber,DAP}. Their
projects used demonstrated the feasibility and utility of using
multitouch surfaces in support of Agile development. Our aim was to
explore some design alternatives based on consideration of the needs
for supporting flexible and lightweight collaboration. The Calgary
work used the Smart Technologies Smart Board and Smart Table and their
Microsoft Windows specific APIs. Our approach was to use open source
frameworks, in particular Python PyMT \cite{PyMt}, a framework
designed for the rapid development of multitouch UI prototypes and the
Community Core Vision (CCV) optical touch infrastructure, and our
software works on a range of hardware and operating systems.


\section{Background}

\subsection{Effectiveness of Agile Storycards and Storywalls}

Our work concerns on user stories, story cards, and cardwalls, and how
they are used by Agile teams. Of particular interest is their use by
the two most common lightweight Agile processes, Extreme Programming
\cite{XPvol} and Scrum \cite{Scrum}. Both of these methodologies take
an iterative approach where less focus is put on upfront analysis and
more on quick consistent deployment of quality working software.  We
intend our work to apply to both XP and Scrum, are we used terminology
from both.

Customers or product owners use storycards to record user stories
which describe features and requirements.  But the card is only a
token that represents a practice.  Jeffries describes the procedure as
the combination of the three C's, Card, Conversation and Confirmation
\cite{Jeffries}. The storycard is deliberately small, preventing
unnecessary verbose detail: Davies says ``The Card may be the most
visible manifestation of a user story, but it is not the most
important'' \cite{PowerStories}.  She continues to explain that cards
``represent customer requirements rather than document them''. This
emphasizes that the actual text on the card is simply a reminder or
placeholder; as Cohn says, ``the details are worked out in the
Conversation and recorded in the Confirmation''
\cite{UserStoriesApplied}. In the process of creating stories the
following should be considered, ``Words, especially when written, are
a very thin medium through which to express requirements for something
as complex as software. With their ability to be misinterpreted we
need to replace written words with frequent conversations between
developers, customers, and users. User stories provide us with a way
of having just enough written down that we don't forget and that we
can estimate and plan while also encouraging this time of
communication'' \cite{UserStoriesApplied}.

The practice starts with the assignment of a story to a developer; as
Sharp et al. say: ``the physical possession of this card by a
developer is a warrant that secures the conversation (and the
confirmation process of them acceptance test) with a customer''
\cite{Sharp}. The subsequent communication between developer and
customer explores the details of the story and the confirmation,
should be mutually agreed upon, such that the story's completion is
well understood.  The cardwall is a tool where the storycards are
organized and displayed.  Typically new stories are placed in the
project backlog and remain their until they are scheduled for
development at which point they are removed from the backlog and
placed on the cardwall along with other stories also scheduled for
development. The grouping and or placement of stories on the cardwall
provides a visual cue as to the state of the stories. A natural
question might be to wonder how does this seemingly simply, low
technology solution helps software development teams to meet their
goals and deadlines while producing quality code? Often, it is the
human component that is the key to the success of the method. Any
software trying to replace a physical task must consider this issue in
depth and try and provide an environment, which supports and
encourages those same interactions, which bring success to the
physical task. 

Sharp et. al \cite{Sharp} strongly suggest that anyone considering
technology to support card and cardwall practices must take account of
the complex relationships that exist within this social system if they
wish to retain key properties of successful teams.  The following is a
summary of some of their key points.

While a general template for stories usually exists such that key
information is usually present like the {\sc As A<Role>}, {\sc I want
  <Description>}, and {\sc So that <Benefit>}, the process is
extremely flexible and we commonly find distinct notation across Agile
teams. At the same time, within any one team, people strictly adhere
to their agreed understanding of the notation and use of
cards. Everything on the card has meaning which is not necessarily
clear to an observer unfamiliar with the team specific notation.  The
location of, color, size of lettering and any annotations carry
significant meaning and thus provide a high level of abstraction. One
of the most compelling aspects of storycard is the flexibility it
affords teams to personalized notation in a manner that works best for
them. 

The use of the wall is also an extremely flexible procedure but, has
it's generalities in that, teams use walls for the duration of a
project and leave them on constant display somewhere they are easily
seen, usually, in a common space where stakeholders can quickly access
key factors like the progress of the project. The wall is generally
regarded as an ``Information Radiator'' \cite{InfoRad} and helps
ensure the transparency of the project. Again, the wall is full of
meaning not obvious to an observer who lacks familiarity with Agile
methodologies or team-specific notation. Also again, the walls have
inconsistent structure across distinct teams, but are used in
extremely consistent manners within any one team.  Maybe more
important still are the social interactions involved in the whole
process; enabling teams to determine their best use of notation,
annotation and layout. These interactions reveal the importance and
meaning of the stories and thus drive their physical placement on the
wall, which, in turn radiates information about each story's progress.

\subsection{Multitouch Tool Support for Agile Development}

There are a limited number of software design tools developed to
harness the new possibilities in Human Computer Interaction afforded
by the current technology in multitouch enabled devices. Everyday more
devices are being produced at a reasonable cost with support for two
or more simultaneous touches; a critical feature for the development
of truly collaborative tools. The best example of similar software is
the Agile Planner for Digital Tabletop (APDT), \cite{Wang}.  APDT was
designed based on a prototype by Weber \cite{Webber} which was
intended for co-located collaboration on a single touch surface. APDT
chose to use this as a starting point but wanted to enhance it with
support for multitouch, the ability to interface with other Agile
planning tools and real world evaluation based on user studies;
observing traditional Agile planning meetings as well as observing
meetings conducted using DAP or Distributed AgilePlanner\cite{DAP} As
the name suggests DAP was designed to support distributed Agile teams
in the planning and maintenance of an Agile project through the use of
a digital whiteboard and storycards. DAP had been developed with a
traditional single user interface paradigm (one keyboard, one mouse)
such that users could collaborate from a distance but not so much in a
co-located environment. APDT also studied and drew from the literature
available on the use of multitouch tabletops in a group
collaboration. APDT was developed as a multitouch enabled tool,
specifically for two tables designed by Smart Technologies Ltd
(Smart), using Smart's proprietary SDK. The first table used DViT
(Digital Vision Touch) \cite{DVT} technology and had support for two
concurrent touches. The second table used FTIR (Frustrated Total
Internal Reflection) \cite{FTIR} technology and had support for 40
concurrent touches. The two touch capabilities of the DViT table
limited the user's ability to work concurrently while the small form
factor of the FTIR table meant that it was difficult to leverage its
support for 40 simultaneous touches. APDT was a highly functional full
featured tool, however, its dependence on the Smart SDK and therefore
on Smart's hardware helped us make a key design decision; we wanted
CMAP to be designed such that hardware and operating system
independence was a goal, as well as support for multiple concurrent
touches.

\subsection{Collaboration in Digital Workspaces}

Scott, Grant and Mandryk have considered studies of digital
workspaces, and identified guidelines for co-located collaborative
work on a large interactive displays \cite{ScottGuidelines}.  They
suggest there are eight key elements that must be addressed via the
physical hardware of the tabletop, via the software being used on the
tabletop or by a combination of the two. The following is a summary of
those key requirements and a brief description of each.


\begin{description}
\item[Support interpersonal interaction] The technology must support
  the mediation of the collaborative interaction and must not
  interfere with this interaction. Ideally it should be as natural to
  collaborate around a digital tabletop as it is to collaborate around
  a regular table.

\item[Support for fluid Transitions between Activities] Switching
  tasks during collaboration should be as seamless as possible. For
  example, if the activity needs to combine data entry and the ability
  to draw, then switching between these activities must be a natural
  process. This allows the focus to remain on communication. The use
  of multiple input mechanisms must be considered but, should be a
  feature that enhances the overall activity and not a hindrance for
  these transitions.

\item[Support for Transitions between Personal and Group Work] If the
  collaborative task involves a combination of both personal and group
  work, the system should try and capture this by providing a similar
  mechanism. The physical shape of the table might be a key factor for
  this point because the individuals must feel comfortable and their
  personal workspace must not feel cramped or invaded. One suggestion
  may be to support external devices or displays that could be used in
  conjunction with the digital tabletop. This approach may however,
  interfere with the fluid collaboration Scott suggests that this area
  needs more study.

\item[Support Transitions between Tabletop Collaboration and External
  Work] It should be easy to integrate previous work into the shared
  environment where it can viewed and manipulated as needed.

\item[Support the use of Physical Objects] The system should support
  both work and non-work related physical artifacts. For example,
  there should be no negative impact on the system if a non-work
  related object were placed on the table while the same action using
  a digital work related object might cause a communication link to be
  established, giving access to this work-related object.

\item[Provide Shared Access to Physical and Digital Objects] While
  collaborating around a traditional table, pointing and other
  gestures are usually easily interpreted by the group. On a digital
  surface this may not always be the case. A digital tabletop may
  provide several representations of the shared object (maybe one for
  each person) using gestures to try and point something out in this
  scenario could lead to confusion as users try to interpret how the
  gesture related to the object they are viewing. If however only one
  representation of the object is present, those same gestures may be
  a contributing factor to the overall understanding and group
  collaboration. The digital representation of shared objects is a key
  to the successfulness of the collaborative session. The designers
  must consider the potential physical locations of the participants
  as well as the possibility of obstruction by other users, objects or
  even gestures. Obvious examples of physical objects are digital
  artefacts like an IPad. A non-digital artefact might be the pieces
  or tokens of a board game implemented for the tabletop where users
  interact with the pieces in game-play.

\item[Considerations for the Appropriate Arrangements of Users] The
  system must be designed to accommodate sufficient personnel space to
  allow users to comfortably interact with each other without feeling
  cramped. Also, the intended audience is important because the
  physical difference between adults and children suggest that more
  children could comfortably interact around a table. Children also
  tend to want to be closer to their neighbours then most
  adults\cite{Aiello}. The system should also not be affected by
  participants repositioning themselves around the table. Virtual
  objects should be rotatable thus allowing equitable user interaction
  from any position.

\item[Support Simultaneous User Actions] This is an area where true
  hardware and software advancements have developed since the
  publication of Scott's paper in 2003. Most current tabletop
  implementations support both multiple input devices and concurrent
  user input.
\end{description}
