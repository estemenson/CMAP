\documentclass{sig-alternate}
\usepackage{url}
\begin{document}
\title{A Literature Review of Visual Formalisms}
\numberofauthors{1}
\author{ \alignauthor
Stevenson Gossage\\
\affaddr{School of Computer Science, Carleton University}\\
\affaddr{1125 Colonel By Drive, Ottawa, ON K1S 5B6}\\
\affaddr{Ottawa, Canada}\\
\email{sgossage@connect.carleton.ca}
}
\date{21 April 2011}


\maketitle

\terms{Visual Formalism, Visualization Evaluation, Semanatic
Interface, Distributed Collaboration}


\keywords{Visualization, Formalism, Collaboration}

\section{Introduction}
We have been representing information in a visual manner for centuries, long
before computers had ever even been imagined. Pictures and drawings were
probably the first methods of visually representing information followed by the
written word which started to appear between 4000 and 3000 BC
\cite{HistoryOfWriting}. Maps and charts were being used as early as 2300 BC
\cite{HistoryMapsCartography}. Today we are still using physical media to
visually represent and communicate information, from graphs, maps, calendars,
sketches etc. With the introduction of computers and especially with the
advances in graphical drawing technologies we have reached a point where visual
information representation can go far beyond any affordance of a tradition
physical media like paper. We now have the ability to not just to create static
visualizations, but rich, active visualizations with many interesting properties
like sharing and mass distribution, versioning, history, dynamic updates, and
support for manipulation. These are just a few of the advantages that can be
leveraged by presenting information graphically. Formally, a visual formalism is
simply a traditional visual notation like the maps and graphs mention above but
designed for the computer. ``Visual formalisms are diagrammatic notations with
well-defined semantics for expressing relations. They are based on simple visual
notations such as tables, graph, plots, panels and maps" \cite{Nardi:BeyondModels}.
\subsection{Motivation}
The interest in visual formalisms stems from the direct implications they have
for my master thesis project. I am interested in leveraging a table visual
formalism to create an agile story card wall. In general, a card wall consists
of stories and task, each of which are represented on small cue cards (or some
similar medium) and organized into columns and rows on a wall. The table
formalism will help put the focus on the structure and allowable manipulations
of the cells and there presentations. The card wall  a specialized
implementation of the table formalism will define the contents of the cells
(stories and tasks) and the operations that make sense within the confines and
constraints of the table.
\section{Summary of the reviewed Papers}
\subsection{Summary: Information Visualization\cite{Gershon:1998:Informationvisualization}}
This paper explains that visualization is a technique used to display
information in a graphical or visual manner with the intent of leveraging our
natural visual capabilities. Visualization allows us to process, analyze and
understand massive amounts of data very quickly. The problem with visualizations
is that we are trying to use them to quickly understand more and more abstract
data that does not always have an immediately obvious mapping to the real world.
This presents a challenge in trying to develop reusable meaningful
visualizations. There is ongoing research to try and leverage metaphors and to
understand what types of analytic tasks could be enhance by visualizations based
on those metaphors.

The paper talks about visualization developments in several areas including,
document and software visualization, distortion techniques, hierarchies, data
integration (including data warehousing and data mining) and the world wide web.
At the time of publication of this paper in the late 90's there were also some
real world commercial applications including applications based on work done at
Xerox PARC such as Perspective Wall, Cone Tree and Wide Widgets.

The author suggests that, at the time, there was still much work to be done in
terms of understanding the potential and usefulness of the technology but it was
clear that visualization would play a huge role in the future of computing.
Gershon describes visualization as an emerging discipline which must progress
through four fundamental stages and suggests that visualization is progressing
at an accelerated pace because many of the step are happening in parallel; ``An
emerging discipline progresses through four stages. First, it starts as a craft,
practiced by artisans using heuristics. Second, researchers formulate scientific
principles and theories to gain insights about the processes. Third, engineers
eventually refine these principles and insights into production rules. Fourth,
the technology becomes widely available.'' \cite{Gershon:1998:Informationvisualization}
 
 
The paper suggests that widespread usefulness of visualizations will be directly
influenced by the content providers' (software design and developers) ability to
create visualizations that are accessible and easily understood by a wide range
of end users, making them equally useful for users of diverse backgrounds. In
order to create these meaningful visualizations in real world applications the
systems being modeled and the processes involved must be understood. This
understanding in itself could lead to the correct visualization for any
particular task.
 
 
There are many areas that we need to learn more about in order to better support
human learning through visualizations such as what types of interactions make
sense for any particular visualization, how we perceive and understand
information both visually and non-visually as well as how we search for
information and how we adapt those searches based on previous knowledge. A
better understanding of these issues and how they relate to specific analytic
tasks will inform the creation of more useful, meaningful visualizations that
that can truly help end users process more data more accurately and quickly. The
ability to create more flexible task oriented user interfaces can also aid in
the development of more useful visualizations.
 
 
An important point that Gershon makes in the paper is that we also need to
understand when graphic visualization may, in fact, be less appropriate.
Sometimes visualizations may be harder to understand than words; we need to
weigh the options and choose the representation that is most appropriate to
allow the users to accomplish the required task.
 
 
In conclusion the paper re-iterates that there is great need for the evolution
of this emerging technology especially in ``The development of scientific and
engineering principles for the generation of visualizations  (to users with
diverse needs and capabilities) and a methodology for solving problems with
information visualizations are badly needed.'' \cite{Gershon:1998:Informationvisualization}

\subsection{Summary: Situational Awareness Support to Enhance Teamwork in
Collaborative Environments \cite{Kulyk:2008:SituationalAwareness}} This paper
was motivated by that fact that today, many modern teams are collaborating using
a multitude of visualizations on multiple large format displays. During these
activities the situational awareness of the teams is not always ideal. This
paper attempts to develop some guidelines to help design better co-located
collaborative systems for use with multiple large format displays with
affordance for situational awareness.
 
 
There were several concepts developed to help support situational awareness on
large shared displays including an interface the authors described as ``Memory
Board is an interface that automatically stores and visualizes the activity
history'' \cite{Kulyk:2008:SituationalAwareness}. It provides participants access to historical annotations
made on other views or visualizations and allows them to be aware of which member is
controlling a particular visualization or display. They also developed the
\emph{Highlighting on Demand} interface which allows individuals to highlight or
fade out any part of any display using a personal interaction device.
 
 
The paper argues that simply providing complex visualization across or using
multiple displays may, in fact, increase the cognitive load and have a negative
impact on the collaborative activity. Designing situational awareness into these
applications is necessary to properly coordinate team decision making and truly
support the collaborative process. One reason is that the participants can be
easily distracted by sub-tasks that should in fact be transparent to the
activity but in reality are not, like who is in control of a shared artefact or
which display or visualization is the one currently of interest. Shared
awareness of these types of interactions are essential and can ``leads to
informal social interactions and development of shared working cultures which
are essential aspects of group cohesion''
\cite{Kulyk:2008:SituationalAwareness}.
 
 
Kulyk states that situated awareness is concerned with the individual's
knowledge and understanding of the events, information and environment as well
as the shared understanding of the team as a whole and their understanding of
the past and present and even its impact on future events. Kulyk's work focuses
on the following three main aspects which directly relate to and extend
Endsley's \cite{Endsley:1988:DesignEvaluationSituationAwarenessEnhancement} three levels of situated awareness:
\begin{itemize}
\item A person's previous knowledge and understanding which includes the
source and nature of relevant events and information.
\item Detection and comprehension of the relevant perceptual cues and
information from the environment including the comprehension of multiple
visualizations and their context.
\item Interpretation of theses visualizations and the continuous reconfiguring
of understanding and knowledge during collaboration. This supports the
awareness of changes in the environment as well as the ability to keep track
of work in progress.\end{itemize} Kulyk is actually interested in the impact
of situational awareness on team collaboration and specifically what she
refers to as \emph{shared situational awareness}. She defines it as the level
of the group or team's awareness of individual situational awareness based
on the above three points, ```the degree to which every team member possesses the
situation awareness required for his or her responsibilities'' \cite{Endsley:2000:SituationAwarenessAircraftMaintenance}. With
respect to shared situational awareness, the paper is mainly concerned with
answering the following questions:
\begin{itemize}
\item Understand the impact of situational awareness on
collaboration.
\item Understand how to support shared situational awareness in collaborative
environments.
\item Understand how to leverage the usage of shared large format monitors to
support shared situational awareness.
\item How can new interactive systems and visualizations be designed and
evaluated based on their support for situational awareness and how might these
systems actually stimulate new and existing forms of collaboration.
\end{itemize}
The paper points outs that the level of support for situational awareness
varies depending on the collaborative activity. For example a collaborative team
that supports emergency dispatch would need a higher level of support when
compared to a team working on scientific collaboration. In the long run,
whatever the reason for the collaboration, mistakes cost and better situational
awareness support can help reduce mistakes.

Next the paper talks specifically about situational awareness and scientific
collaboration and in particular how large shared displays could be used to
better support situational awareness in current omics experimentation in
molecular biology. The following key points were made:
\begin{itemize}
\item Visualizations on a shared display encourages group discussions.
\item The visualization of data on a shared display allows users to quickly
determine the quality of the information or data-set.
\item Multiple visualizations may be needed and when used there must be a
clear visual relationship between these visualizations so as to avoid
participants from getting lost, from being distracted or from being inflicted
with change blindness which occurs when the participants do not realize that the
focus of the discussion has changed.

\item When multiple shared displays are being used with multiple co-related
visualizations, changes in a particular visualization should also trigger
changes in the other related visualizations.
\item Difficulties arise when team members from diverse disciplines need to
use the same visualizations that may have been designed to be understood by
people with a specific background or skill-set as is the case in many
scientific visualizations.
\end{itemize}
To address some of the problems related to situational awareness for shared
displays the authors implemented Highlighting on demand interface which allowed
the person currently controlling the shared display to highlight or fade out
certain visualizations or areas of the shared display(s). This increased
awareness among the team as to the relevant visualization at that moment. A
second solution was the memory board interface which stores and visualizes the
changes in real-time during team collaboration. This ensures that the entire
collaborative process is recorded and all intermediate visualizations,
annotations etc. can be retrieved and reviewed. A control interface is
suggested as a means of controlling the shared displays and visualizations and
giving access information about the visualizations and displays. This
was envisioned to run on a shared touch screen as well as on the team members
individual interaction devices. Knowing who is making changes and what changes
are being made is critical to any collaborative effort and the control interface
attempt to address this problem.
\subsection{Summary: Beyond Models and Metaphors: Visual Formalisms in User
Interface Design\cite{Nardi:BeyondModels}}
In this paper Nardi argues that the user interface can be designed to support
the visual representation of the underlying program's semantics and can help
offload cognitive load to facilitate problem solving. Nardi postulates that at
the time this paper was published in the early 90's it was time to start
focusing more attention on the semantic interface rather than the also
important but less interesting syntactic interface design to relay user
commands to the underlying application. She claims that design and analysis tasks are complex and software designed to support these activities needs to be enhanced with well-designed
visual interfaces from which the semantics of the application could be easily
inferred by the users. The designers of this type of complex application would
be well served to have a toolkit of computational building blocks, such that
they could piece together the appropriate blocks which could support this
visual transfer of application semantics.
 
 
Nardi argues that neither Mental models nor Metaphors are appropriate
theoretical foundations to support such  building blocks but that these blocks
could be developed by leveraging the power of visual formalisms. She argues that
different visual formalisms can be used as the building blocks for semantic
interfaces and showed how a table visual formalism was implemented and explains
the potential for it to be used as a building block. Nardi defines visual
formalisms as objects that have their own inherent semantics and don't need to
borrow the semantics from some other domains for users to understand them,
``They are based on simple visual notations such as tables, graph, plots, panels
and maps'' \cite{Nardi:BeyondModels}. Visual formalism can be specialized or
extended to fit a particular problem or application. An example is the table
formalism which has been tailor fit into applications such as spreadsheets,
calendars, schedules etc.  Nardi recognizes the potential for computer based
visual formalism to support interaction, modification, relationships  to other
components, notation mapping including handwriting to text conversions etc.
Nardi attempts to compare how mental models, metaphors and visual formalisms
support the design and development of rich applications that convey the
semantics of the application through the user interface and how they support
application development ``through reusable semantic objects that support
application development at a higher level than general programming languages.''
\cite{Nardi:BeyondModels}. 
 
 
In particular the paper is interested in the design
of semantic user interfaces for software that supports complex, highly
collaborative design and analysis, with explicit support for creative problem
solving within the application. Nardi is quick to point out that the usefulness
of mental models and metaphors are not being questioned in general, it is their
usefulness for the specific task of creating collaborative software that
supports a semantic user interface and complex problem solving that is being
questioned. Nardi argues that visual formalism provide the correct level of
abstraction to create a library of reusable extensible objects that can truly
help the application developers of such systems. 
 
 
The paper argues that many
studies of mental models have shown results which lack support for their ease of
use, and for our ability to use them to help with complex cognitive tasks. In
fact there is little agreement among psychologist as to their exact definition
or even of their existence. The paper uses an example of one experiment where
participants found it difficult to perform seemingly simple cognitive tasks
based on a mental image, a concept even simpler then mental models. To
illustrate participants were asked to imagine a tiger, and, were then asked to
count the stripes of the tiger, a task that most participants found difficult or
impossible \cite{Reisberg:ExternalRepresentations}.  Other experiments by
Reisberg also supported the apparent difficulty of accomplishing other simple
tasks based on a clear and simple mental image. Reisberg concludes that mental images are less complete and
more rigid then we imagine. Nardi provides support for the argument that if these
simple tasks are difficult and the mental images are rigid and incomplete then
how will mental models provide adequate support for more complex, cognitively
intense systems. Nardi cites work by Kintsch
\cite{Kintsch:88:TheRoleOfKnowledge} who suggests a more flexible system, where the structure is generated in the context of the task and rules and propositions
are composed based on context. Nardi argues that this more flexible strategy is
what is needed for the complex systems under discussion. The work of Rips
\cite{Rips:MentalMuddles} is also referenced as evidence for the indiscriminate
use of mental models. In many cases propositional or production rules derived by
domain knowledge is simply re-cast as mental models, when they would be better
off to be thought of as production rules. The paper argues that to match a
mental model to a semantic interface would mean to devise a mental model that
was ``capable of running mental simulations (exactly as Hollan and his
colleagues proposed)'' \cite{Nardi:BeyondModels}. Nardi doubts if this type of
complex mental even exists. The paper suggests that the task placed on
developers to understand the mental models and implement them in a semantic
interface is overwhelming  since, ``This approach gives no tools to designers,
but asks them to go out and do what professional cognitive psychologists have
been unable to do - convincingly describe meaningful mental models of any but
the simplest phenomena.'' \cite{Nardi:BeyondModels}. Nardi also argues against
the views of Hutchins \cite{Hutchins:1985:DirectManipulationIinterfaces} that
suggests that the user interface and the tasks' must be matched with the user's
mental model of those tasks and that the interaction with the user interface
should directly correspond to the user's mental model. In this way reducing
cognitive load because thoughts are easily translated into the actions needed to
be performed on the interface to accomplish the goal. Nardi explains that this
type of view limits the interface to a simple means to an ends instead of seeing
it as having the potential to support cognitive offloading by providing the
necessary visualizations for quick and accurate transference of semantic
information about the system and the tasks at hand. It can help guide the user
and support them in the processing of complex problem solving; ``it is not a
passive receptacle for thoughts emanating from an internal model, but plays an active
role in the problem solving process'' \cite{Nardi:BeyondModels}. 
 
 
An ethnographic
study was performed to try and understand how people use spreadsheets as
computational devices to solve real world problems. The results of the study
gave support for the idea that people do not tend to start from mental models,
``but instead incrementally develop external, physical models of their
problems'' \cite{Nardi:BeyondModels}. It was found that users tend to start off
with unclear goals and through the use of the spreadsheet they are able to
organize their data and start to see relationships which lead to the refinement
of their problem set and ultimately to the solution; ``the representation of the
problem emerges through the medium of the spreadsheet. The representation is not
``in the user's head,'' nor was it ever in the head, but becomes an artefact
created by the user's interactions with the program.''
\cite{Nardi:BeyondModels}. Nardi states that if the users are not using mental
models as a  means to solve their problems then how can developers design
interfaces based on the mental models of the users? Tools like the spreadsheet
leverage a table formalism to support the user's own development of an external
model of their problem set. Users are able to create personalized models of
their explicit problem within the constraints of the tool and guided by the
affordances of said tool. Nardi suggests that focusing on mental models
distracts us from understanding user activities in terms of (1) high level task
goals and (2) the data structures and  the operations that need to be supported
to allow users to effectively model their unique problems. This evolutionary
modeling results in an externalized model that can be used to leverage what is referred to by Reisberg \cite{Reisberg:ExternalRepresentations} as perceptual knowledge and
described it as knowledge that can only be accessed  through interaction of
external representations. This means that there are some forms of knowledge,
that can only be accessed through visual cues. Nardi believes that it is more
important to understand how external representations or models aide us in performing complex
cognitive tasks and how we can build semantic interfaces to support the users'
ability to create and interpret these external visualizations using visual
formalisms as the building blocks.
 
 
Next the paper attempts to provide support for the argument that metaphors are
also ``unsuitable for expressing rich application semantics, and inappropriate
for the reusable computational structures we seek'' \cite{Nardi:BeyondModels}.
Nardi states that the simple, once useful metaphors of the desktop, Rolodex, or
trash can icons are no longer needed to help end users understand how to
interact with their computers and applications. The user interface has also moved beyond
simply relaying user commands to the underlying system, it now crucial to
understand how the interface can convey rich application semantics and provide
support for complex cognitive operations. Metaphors lack precision and are
incomplete representations of any but the most simple of applications;
``Metaphors are good at suggesting a general orientation, but not good at
accurately encoding precise semantics'' \cite{Nardi:BeyondModels}, but it is
precisely those semantics that the interfaces under discussion are trying to
capture, convey and support. Another problem is that metaphors are not reusable
and not extensible and become stale, and therefore could not be used to support
developers in creating these types of applications. Carroll and Thomas
\cite{Carroll:MetaphorAndTheCognitiveRepresentation} suggest eight rules to help
develop or recognize good metaphors, the first one states that metaphors are
system dependent and must also account for the expected user of the system. This
alone ties metaphors to the systems they represent and gives evidence for their
lack of re-usability and extension. Also, metaphors, become stale, ``However,
for most computer systems there will come a point at which the metaphor or
metaphors that initially helped the users understand the system will begin to
hinder further learning... When the original metaphor begins to fail, either
new, more detailed metaphors will have to be introduced or the metaphorical
approach will have to be abandoned in favour of a more literal
approach.''\cite{Carroll:MetaphorAndTheCognitiveRepresentation}. 
 
 
Nardi also
talks about the impact of prior knowledge and how it can sometimes temp us into
seeing and using metaphors which simply are no there. As an example Carroll
\cite{Carroll:1988:InterfaceMetaphorsAndUserInterfaceDesign} suggests that a
ledger sheet is the metaphor representing spreadsheets, Nardi argues that in
fact spreadsheets are simply tables and that there are no metaphors that can
describe the generality and expressive power of the table. At best metaphors
could be used to express specific uses of tables, but nothing that could come
close to capturing the entire possible semantics. Nardi argues that it  is far
too easy to ``confuse any kind of prior knowledge with metaphor ''
\cite{Nardi:BeyondModels}. 

Next the paper turn its attention of explain why visual formalisms are the
appropriate method to help designer and developer convey application semantics through the user interface.
Harel  clearly expresses the need to support complex cognitive activities and
problem solving through semantic interfaces when he said, ``They will be
designed to encourage visual modes of thinking when tackling systems of
ever-increasing complexity'' \cite{Harel:1988:VisualFormalisms}. We use physical
or real visual formalisms all the time in everyday life, including maps, tables,
sketches and graphs; digitized visual formalisms bring a new dimension
previously unavailable. Notably, computerized versions can be stored, search, retrieved, versioned, they also afford virtually instant sharing (or even
real-time sharing) and can support modification by collaborators near and far
(with whom the visualizations were shared). These visual formalisms must provide
support for the user to be able to draw meaning not just from viewing them in a
static manner, rather, by the ability to manipulate, compose, add or subtract
them, search, filter and combine them in ways that make sense for their
particular application.  Implementing these allows our designers and developers
the opportunity to use them as building blocks to create complex semantic
interfaces that support hard cognitive processes through visual guidance and
visual support of semantic reasoning. Nardi suggests that visual formalisms are
application frameworks because they provide the structure in which to support a
specialized problem. As an example, the spreadsheet is a specialization of a
table formalism, the table formalism provides the underlying base functionality
of the system, but the specialized implementation (the spreadsheet in this case)
provides the structure and constraints that allow the end user to focus on their
dataset and allow the problem to emerge in the form of an external model,
created and guided by, and within the boundaries of those constraints. In this
way ``The initial phase of a modeling problem is reduced to simply recognizing a
format into which a problem is framed, rather than being faced with the
necessity of inventing a format from scratch''
\cite{Nardi:90:TheSpreadsheetInterface}. Nardi describes the strengths and
properties of visual formalism in the following list:

\begin{itemize}
\item \emph{Exploitation of human visual skills.} Visual formalisms are based
on human visual abilities, such as detecting linear patterns or enclosure,
that people perform almost effortlessly. Visual formalisms take advantage of
our ability to perceive spatial relationships and to infer structure and
meaning from those relationships (Cleveland, 1990). Visual formalisms are
capable of showing a large quantity of data in a small space, and of providing
unambiguous semantic information about the relations between the data. \item
\emph{Strong semantics.} Of the visual formalisms we have identified, graphs
have defined for them the most formal semantics, and panels the least. Any
implementation of a visual notation as a visual formalism will supply its own
formal semantics. A computer-based version of a panel may only specify that a
panel is an arbitrary collection of objects that belong together. Its strength
as a presentation format makes it an extremely useful object, and an
implementation might have a great deal of presentation functionality in
addition to the semantics of being able to handle a collection of objects
correctly for a given application. \item \emph{Manipulability.}  Visual
formalisms are not static displays, but allow users to access and manipulate
the displays and their contents in ways appropriate to the application in
which they are used. \item \emph{Specializability.}  Visual formalisms provide
basic objects that can be specialized to meet the needs of specific
applications. They are at the right level of granularity neither too specific
nor too general. Visual formalisms are appropriately positioned between the
expressivity of general programming languages, and the particular semantics of
applications. \item \emph{Broad applicability.}  Visual formalisms are useful
because they express a fairly generic set of semantic relations, relevant to a
wide range of application domains. Because a large number of applications can
be designed around a given formalism, visual formalisms will eliminate a great
deal of tedious low-level programming, as well as give developers ideas about
editing and browsing techniques with which they may not familiar, such as the
use of fish-eye views for large datasets
\cite{Furnas:1986:GeneralizedFisheyeViews}\cite{Ciccarelli:1990:BrowsingSchematics}.
\item \emph{Familiarity.}
Because the standard visual notations are so useful, they are found
everywhere. Not only do they draw on innate perceptual abilities, but through
constant exposure we become very familiar with them. Our schooling explicitly
trains us in the use of the basic notations; e.g., using calendars and
learning matrix algebra provides experience with tables. Everyday activities
provide opportunities to create and use visual notations, such as writing a
laundry list or reading a map.\end{itemize} Finally to wrap things up, the
paper discusses their implementation of a table visual formalism as an
illustration of how a visual formalism (or more generally a set of them) can
be used to help developers build rich semantic interfaces for systems designed
to support complex cognitive tasks including
problem solving. 
 
 
To start, the implementation of any visual formalism must
begin with by defining the semantics of the visual formalism. This involves
identifying how individual components are structured, how they relate to each
other and what operations need to be supported. The presentation and layout as a
whole, and as individual components, must be considered and formalized. The
application semantics within the constraints of the visual formalism must be
clearly understood. This is really defining how the framework can express the
underlying semantics of the system. Their implementation creates a more flexible
table then you might find in a typical spreadsheet application but in keeping
with the proclaimed advantages, it could be used to create a
spreadsheet. The table formalism is a component that can contain other
components (cells). Each cell is disjoint but can be set to span rows or columns
as well as be merged with other cells or split into multiple cells. Cells can
also be added (created), deleted, moved, copied and pasted. There is support for
labeling rows and column as well as naming regions ( a group of cells). The
ability to iterate over cells or regions, to scroll, both horizontally and
vertically as well as support for zooming in and out cells, rows, columns or
regions. There is support for filtering (essentially temporarily hides certain
cells, regions, rows or columns). Another big difference of the table formalism
is that the cells can contain anything from a primitive to a complex class or
widget or even an application. 

To finish the paper provided strong support for
the use of visual formalisms to aide developers in the design of complex
systems. It clearly illustrates where mental models and metaphors fall short as
a theoretical foundation for the creation of these types of systems. Finally it
gave a good concrete description of a table visual formalism.
\subsection{Summary: Heuristics for Information Visualization Evaluation
\cite{Zuk:2006:HeuristicsForInformationVisualizationEvaluation}} This paper
defines heuristic evaluation as a quick and lightweight method to predict and
spot issues related to the usability of an applications' user interface. The
heuristic approach involves a small group, testing the system based on the
heuristics or guidelines which are relevant to the system. Heuristics can often
be used to teach novice users and as a method of identifying design patterns.
Like design patterns, heuristics help by providing an overarching language from
which the system can discussed and they promote reuse especially through the
emerging design patterns. Zuk describes heuristic evaluation as commonplace
in the field of HCI but, its application to information visualization is still
emerging.  The author states that we are at a point where there is a need to
create a small set of general  overarching heuristics that could be applied to
most information visualization system, similar to Neilson's 10 usability
heuristics for the design of user interfaces. The paper describes this as an
exciting and open area for future study. Zuk suggests  that a hierarchical
grouping of existing heuristics combined with different tree searching
algorithms could help in selecting a core set of heuristics. Otherwise,  one
could follow Neilson's method of reducing a large set of problems into a small
set of easily understandable heuristics. 

Next the paper describes the process of
heuristic evaluation as something that is apt to evolve but nonetheless
describes that an iterative approach using five evaluators have been found to be
effective. In the first pass the evaluators try and get a general sense of the
application and a second pass is used to focus on applying the heuristics to
specific elements of the user interface. The exact number of evaluators will
most likely be application specific and dependent on the complexity of the
visualizations. The paper refers to work by Tory and Moller
\cite{Tory:2005:EvaluatingVisualizations} who suggest using a combination of
traditional usability experts and visualization or data display experts. It is
still to be determined what the exact characteristics of an information
visualization evaluator should be.  The paper describes work done by Craft and
Cairns \cite{Craft:2005:VisualInformationSeekingMantra} as determining that
there is still much work to be done in this area, ``They conclude by calling for
a more rigorous design methodology that: takes into account the useful
techniques that guidelines and patterns suggest, has measurable validity, is
based upon a user-centered development framework, provides step-by-step
approach, and is useful for both novices and
experts.''\cite{Zuk:2006:HeuristicsForInformationVisualizationEvaluation} The
paper also warns that there could be problems if we blindly apply existing
knowledge and methods of usability heuristics to information visualization and
suggest that caution should be used and further study is needed. Zuk noted in
the case study that the evaluators had problems with the heuristics that were
too specific and felt a more general vocabulary would increase its usefulness.
It is was also noted that in general more precise, clear and understandable
names and descriptions of each heuristics would facilitate their use. This would
also help ensure that they are used as intended beause there is less room for
misinterpretation. The case study revealed that some problems were detected by distinct heuristics which the author suggested could be used as way to see the problem from distinct
points of views and possibly could lead to multiple solutions.  It was also
found that assigned priorities could help resolve issues when conflicting
heuristics were found. Another suggestion was to let domain experts resolve
heuristic conflicts. The authors also identified some problems that may be
common among visualizations in general. Some of the issues include:
\begin{itemize} \item Poor contrast can produce hidden or difficult to see
visual components. \item Poor choice of colors could lead to confusion in terms
of understanding or misinterpreting the relationships between visualizations.
\item Tool tips lack the detail required to make them useful.
\end{itemize}

The paper notes that some heuristics would be better used as guidelines for the
designers and or developers of the system while others might be better for
heuristic evaluation of the visualization. Zuk suggests that paired evaluator
teams could be comprised of evaluators paired with domain experts, this could
help in a more accurate application of the heuristics. It was also noted that
usability heuristics could be useful in combination with the visualization
heuristics but, notes that this requires further study. Another open research
area would be to discover if tools could be used to support the evaluation of
visualizations.
\subsection{Summary: A Collaborative Dimensions Framework: Understanding the
Mediating Role of Conceptual Visualizations in Collaborative Knowledge Work
\cite{Bresciani:ACollaborativeDimensionsFramework}} The goal as stated in the
paper ``is to identify the factors that contribute to the choice of an effective
visual representation for collaborative knowledge work to support distributed
cognition, and to organize them in a conceptual framework''.  Bresciani  focuses
on visualizations as artefacts and claims that this focus allows for decision
factors to be expressed in terms of dimension that either support  or interfere
with certain kinds of visualization uses. These dimensions are not independent
and often, supporting one dimension will have a negative impact on the support
for another. Understanding the relationships between dimensions allows designers
of collaborative visualizations to pick and leverage those dimension that make
the most sense for the application being developed. The author suggests  that
this approach will help build better support for collaborative activities then
could be achieved using general guidelines that should apply to all
visualizations as might be the case using a heuristic approach. This paper
leverages research from three main fields including:

\begin{itemize}
\item Blackwell's work on the cognitive dimensions of notations framework
\cite{Blackwell:NotationalSystems} and Hunhausen's work on communicative
dimensions framework \cite{Hundhausen:CommunicativeDimensionsFramework}
\item Star's research on the boundary object paradigm \cite{Star:BoundaryObjects}
\item  Thirdly a literature review on visualization technologies, in
particular the information visualization work of Shneiderman and Karabeg as
well as the knowledge visualization work of Eppler and Suthers
\cite{Eppler:2004:FacilitatingKnowledgeCommunication},\cite{Suthers:2005:CollaborativeKnowledgeConstruction}.

\end{itemize}

Informed by  this theoretical foundation and a combination of techniques used
to gather the experience and knowledge of expert practitioners who employ
visualizations in their daily collaborative activities, a collaborative
visualizations framework was developed to help understand how visualizations can
be leveraged to support collaborative activities and how they can facilitate the
distribution of knowledge across team members. The framework consisted of the
following seven dimensions:

\begin{itemize}
  \item \emph{Visual impact}: The extent to which the visualization is
  attractive and if it facilitates attention and recall. \item \emph{Clarity}:
  Is the visualization self-explanatory, easy to understand and dose it reduce
  cognitive effort. \item \emph{Perceived finishedness (provisionality)}: Is
  the visualization perceived by the user as a final product. \item
  \emph{Directed focus}:Does the visualization direct the user's attention to
  any particular area or areas. \item \emph{Inference support}: Do the
  constraints of the visualization help or  support the creation of new and
  novel insights. ``Inference support is the core differentiator and added
  value of visualization over text: it allows to gain new understanding ``for
  free'' just by changing the visualization type, the focus, or the
  representational constrains.'' \cite{Bresciani:ACollaborativeDimensionsFramework}
  \item \emph{Modifiability}: Is there support for dynamic modification of
  the visualization?
  \item \emph{Discourse management}: Does the visualization support control
  over the discussion and work-flow of the activity.
  \end{itemize}
  The paper applied the dimensions to real world visualizations to determine
  to what extent the different dimensions were supported by the
  visualizations. They identified six different but common forms of
  collaboration and suggested that some dimensions would be more appropriate
  for certain collaborative activities, The following lists each of the six
  collaborative activities along with the dimensions that provide the best
  support for those activities:
  \begin{itemize}
    \item idea generation (high visual impact, low clarity, low perceived
    finishedness, high modifiability) \item general knowledge sharing (high
    visual impact, low perceived finishedness)
\item problem analysis (low perceived finishedness, high clarity, high
inference support) \item option evaluation (high clarity, high directed focus,
high inference support, moderate-low modifiability) \item deliberation or
decision elaboration (high clarity, high directed focus, high inference
support, moderate-low modifiability)

\item planning (high modifiability, low perceived finishedness)
\end{itemize}
As noted the above dimensions are not always independent and careful
consideration must be made when deciding the level of support for each
dimension. This concept of trade-offs between dimensions was first introduced
by Green and Petre \cite{Green:1996:UsabilityAnalysisOfVisualProgrammingEnvironments}.
As an example of a trade-off referred to in the paper in reference to the
fireworks  visualization from Let's Focus collaborative software which was used
as one of  the three visualizations to which they applied the collaborative
dimensions.   The application of the dimensions to the fireworks visualization
revealed several relationships or trade-offs between the dimensions. Visual
impact had effect on clarity and on directed focus. The high visual impact of
the fireworks visualization was distracting and therefor reduced its clarity and
it had reduced support for directed focus because of the appealing nature of the
visualization in fact, ``When the visual stimuli are very high, then the focus
will diminish because the attention is caught more by the aesthetics than the
content''\cite{Bresciani:ACollaborativeDimensionsFramework}. There were also
trade-offs between modifiability and perceived finishedness and between
modifiability and discourse management. The visualization had a high level of
support for discourse management through the Let's Focus software package that
provided the visualization. It also had high perceived finishedness ``because
the diagram resembles a final piece of work instead of a discussion
tool.''\cite{Bresciani:ACollaborativeDimensionsFramework}. If visualizations are
perceived as finished, then it will be less likely that collaborators will feel
like they should contribute or modify them. If there is no tool support for
discourse management then high modifiability usually results in in low discourse
management because coordination of the group becomes more difficult.



Another important consideration is the medium on which the visualization is
delivered.  ``The choice of alternative media such as a whiteboard, paper or
computer-based interaction, strongly affects people's willingness to make
contributions and
collaborate.''\cite{Bresciani:ACollaborativeDimensionsFramework}. If people
feel that modifications of the visualization are permanent then they are less
likely to want to contribute, also the perceived finishedness can be affected
by the medium, a piece of paper tends to be thought of as more provisional
where as a visualization on a computer monitor has more of a sense or
permanence even though the computer based visualization may in fact have high
support for modification.


\section{Discussion}
The papers were quite interesting and, when considered as a whole, reveal
several important points about the design and use of visualizations and the
complex applications designed to leverage them. One of the key points is the
need for re-usable visualizations. We need to understand, when and where
visualizations are appropriate and how they can be designed to support and
convey the semantics of the problem or application. We have seen that through
the development and implementation of visual formalism we can create a library
of re-usable extensible or specializable visualizations that can be used by
developers to create these complex applications. Visualizations give us the
power to build applications designed to leverage our natural human ability to
process and reason about information in a visual manner. At the same time
computerized visualizations allow the end user of these applications to leverage
everything that a computer is naturally good at like graphic display, sorting,
searching, sharing, storing etc.

As noted by Gershon \cite{Gershon:1998:Informationvisualization} the potential
impact and usefulness of computer visualizations was recognized in the very
early stages of this emerging technology and now, twenty years later, we should
be in a better position in terms of understanding how to leverage them
to support complex cognitive processes. I am not sure if we are truly there yet,
it seems at least on the basis of this literature in this review that we are still in
what he described as the second stage of development of an emerging technology.
This is the stage where researchers formulate theories and gain insights about
the technology. Simultaneously we are also in the third stage where
engineers refine and implement the findings and finally we are, at the same time, in the
final stage where visualizations are massively available to the end user.

The work of some of the early practitioners in this area has shown a promising path to follow. Harel truly demonstrates the power of visual formalisms when he describes the graph as a visual formalism,: ``A graph, in its most basic form, is simply a set of
points, or nodes, connected by edges or arcs. It's role is to represent a
(single] set of elements S and some binary relation R on them. The precise
meaning of the relation R is part of the application and has little to do with
the mathematical properties of the graph itself. (Certain restrictions on the
relation R yield special classes of graphs that are of particular interest, such
as ones that are connected, directed, acyclic, planar, or bipartite."
\cite{Harel:1988:VisualFormalisms}. I believe that Nardi provides some really
good ``engineering principles for the generation of visualizations"
\cite{Nardi:BeyondModels} but, this paper has not had the attention that I
believe it merits. One possible reason is that the paper may have come before it's time; we are now in a better position to leverage her work and implement a library of visual formalisms that leverage
current technology in graphics, and collaborative multi-touch surfaces.

The fact
that we are doing more and more team collaboration using software to support the
process implies that the visualizations we create must also
consider situational awareness. Collaborating by using visualizations means that
the visualizations need to support techniques for individuals and groups to be
aware of shared objects, transitions between different visualizations,
and methods of interaction and communication. With the recent emergence of new smart
handheld devices we also need to learn how to interface with these devices and
leverage them as a method of interaction (and perhaps personal identification)
with visualizations intended for collaborative environments. Furthermore large
multi-touch enabled displays can help to solve many of the problems identified
by Kulyk \cite{Kulyk:2008:SituationalAwareness}.

Direct manipulation of visualizations on large shared multi-touch enabled displays will help participants keep track and focus on the
correct visualization or component under discussion. Shneiderman's Direct
Manipulation principles \cite{Shneiderman:1997:DirectManipulation}: (1.)  visibility of the component interest.
(2.) Physical actions replace complex syntax. (3.) Incremental reversible
operations whose effects are immediately visible. These principles can act as
guidelines to help support situational awareness in visualizations. We must be
able to see the visualizations and any components, in order to truly reap the
benefit of the visualization. The direct manipulation of the visualizations and
components must produce immediately visible changes in the visualizations as
well as clear indications of changes in related components and other related
visualizations.

How do we assess the quality of our visualizations? We need
formal methods to help determine if our visualizations are actually delivering on
their promise, i.e. do they successfully support making hard mental operations
easier, semantic reasoning, inference and decision making? Heuristic evaluation
is a feasible and interesting solution to this problem. One interesting point in Zuk's paper is that he explains that through the process of determining the
proper heuristics and through their use other useful things happen like the
emergence of design patterns. Gershon \cite{Gershon:1998:Informationvisualization} also mentions that in the first
stage of development of an emerging technology that pioneers are guided by the
heuristics of the technology. Can these heuristics be used as a starting point
from which a more general set of widely applicable heuristics can be derived and
for evaluations and design purposes? Determining this set is a major point of
interest for Zuk and others. A recent paper by Forsell \cite{Forsell:2010:AnHeuristicSetForEvaluationInInformationVisualization} uses an
empirical approach as described by Zuk and previously by Nielsen \cite{Nielsen:1994:HeuristicEvaluation}
to create a set of ten general heuristics to help in the evaluation of
visualizations. It is widely agreed that one of the most important features of a
heuristic is how easily understood is it. The evaluators must understand the
heuristic for it to be used as intended; to accomplish this there need to make
the names and description as clear and precise as possible. This may lessen the
temptation for evaluators to use the heuristics in a prescriptive manner instead
of as intended descriptively. In a paper comparing the activity theory and
distributed cognition, Halverson describes the importance of names, ``Activity
Theory has named its theoretical constructs well. Even though some names may
conflict with common use of the terms, naming is very powerful - both for
communicative as well as descriptive reasons" \cite{Halverson:2002:ActivityTheory}.



To end the
discussion I would like to emphasize the importance of the collaborative
dimensions framework of Bresciani. It gives a an excellent theoretical
foundation for the design of visualizations intended to support semantic
inference in a collaborative environment. In the rest of this discussion I will
try and explain how the most important points of the reviewed papers relate non
trivially to the seven collaborative dimensions
\cite{Bresciani:ACollaborativeDimensionsFramework} and suggest relationships to
Blackwell's cognitive dimension of notations \cite{Blackwell:NotationalSystems}.
\subsection{Visual impact}
The visual impact is directly influenced by our understanding of how we visually
perceive  and process visual information, Gershon discusses these points in the
paper, Information Visualization. Visual impact can also influence support for
situational awareness; high visual impact can aide in the detection and
perception of changes in the visualization and related components.
\paragraph{ Related Cognitive Dimensions}
(\emph{visibility, secondary notation})
\subsection{Clarity}
How easy it is for users to understand the visualization? Again this relates to
Gershon's point about understanding how we perceive and process visual
information. In terms of situational awareness a high support of clarity may
help the users see and understand the impacts of the modification of one
visualization on other related visualizations.
\paragraph{Related Cognitive Dimensions}
(\emph{role expressiveness, hidden dependencies, abstraction, hard mental
operations, consistency, closeness of mapping, secondary notation, diffuseness})
\subsection{Perceived Finishedness}
Provisionality as used in Cognitive Dimensions was a better name for this
dimension. The more polished the visualization the less likely it is that users
will expect it to be modifiable. At the same time users are disturbed when a
series of modifications are performed until a point where they consider the work
a final product, at this point further modifications can be a source of
frustration. There must be a balance between provisionality, modifiability and
discourse management. Discourse management of course could provide the complete
modification history in a versioned format that can be used to easily switch
between different versions. The ability to finalize modifications or even be
able to switch back and forth between a more polished presentation and a rougher
sketch would be useful. Maybe support for switching between sketch and text
modes, where handwritten annotations could be viewed as is or as text using a
handwriting recognition package.
\paragraph{Related Cognitive Dimensions}
(\emph{provisionality, premature commitment, viscosity})
\subsection{Directed focus}
Visualizations can be designed to direct the user's attention to a specific area
or component of a visualization or, to disperse their focus across multiple
visualizations, displays etc. Support for this can increase situational
awareness and help to avoid change blindness.
\paragraph{Related Cognitive Dimensions}
(\emph{visibility})
\subsection{Inference Support}
The syntactic interface designed to simply relay commands from the user to the
application is important but less interesting then the semantic interface. Nardi
talks about the need for the interface to support the transfer of application
semantics to users. The ability to support complex cognitive tasks, decision
making, idea generation and semantic reasoning is the goal of the types of
visualization discussed in this paper. Dynamic, modifiable visualizations and
semantic interfaces design with affordance for situational awareness for group
collaboration is needed to support inference.
\paragraph{Related Cognitive Dimensions}
(\emph{role expressiveness, hidden dependencies, hard mental operations,
premature commitment, viscosity, error-proneness, abstraction, secondary
notation, consistency})
\subsection{Modifiability}
The ability to annotate and make modifications makes computer based
visualizations very attractive. Direct manipulation of the visualizations or
components affords user a sense of satisfaction and can help with situational
awareness in terms of the users knowing exactly who is doing what and to which
visualization or component.
\paragraph{Related Cognitive Dimensions}
(\emph{viscosity, error-proneness, premature commitment, abstraction, secondary
notation, consistency})

\subsection{Discourse Management}
Situational awareness includes the ability to be aware of the history of a
collaborative session and of the individual contributions of the team members,
this maps directly to discourse management and can be supported by some sort of
versioned control management. The ability as to switch from a more provisional
view to a one with a less polished feel could be supported by this dimension.
This dimension is concerned with providing the ability to  absorb, and develop
ideas in conjunction with manipulation of the visualizations so as to generate a
deeper understanding of the problem. This is similar to the way external
representations evolved into more precise models and better understanding of the
problems when working with the visual formalism of the table using the
spreadsheet specialization.

\paragraph{Related Cognitive Dimensions}
(\emph{progressive evaluation, consistency, hidden dependencies, error-proneness})

\section{Conclusions}
This  report has provided a summary of several papers related by their focus on
visualizations. It has identified key properties of visualizations in general as
well as specifics for their design, implementation and evaluation. Collaborative
dimensions framework was introduced as a theoretic base from which to think
about them and a development framework was described to help build a library of
re-usable visual formalisms for designers and developers to create rich semantic
interfaces for cognitively complex and scientific applications. Finally a
heuristic based approach to evaluation was introduced to measure the usability
of the visualizations.
\subsection{Future Work}
It should be noted that the paper on heuristic evaluation, the concepts
described in the Collaborative Dimensions paper and it's ancestor, the Cognitive
Dimensions paper could all be interpreted as heuristics and the elucidation of
these heuristics spans the boundary between artisans using heuristics and 
investigators trying to build a formal semantically strong model of
visualizations. This latter work has yet to be undertaken, but the search for
better heuristics is helping to map the territory such that subsequent
investigators will have a substantial and stable starting point. Research could
be done to relate the above papers and the work of Nardi on visual formalisms
to create an overarching, coherent theoretical foundation from which to attack
future work on complex collaborative visualizations.

\bibliographystyle{abbrv}
\bibliography{litReview}
\end{document}




