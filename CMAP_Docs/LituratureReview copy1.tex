\documentclass{sig-alternate}
\begin{document}
\title{Visual Formalism, Liturature Review}
\numberofauthors{1}
\author{ \alignauthor
Stevenson Gossage\\
       \affaddr{School of Computer Science, Carleton University}\\
       \affaddr{1125 Colonel By Drive, Ottawa, ON K1S 5B6}\\
       \affaddr{Ottawa, Canada}\\
       \email{sgossage@connect.carleton.ca}
}
\date{21 April 2011}

\maketitle

\terms{Visual Formalism, Distributed Collaboration}

\keywords{Visual Formalism, Distributed Collaboration}

\section{Introduction}
\subsection{Motivation}
\section{Summary of the reviewed Papers}
\subsection{Summary: Information Visualization\cite{Gershon:1998:Informationvisualization}}
This paper explains that a visualization is a technique used to display
information in a graphical or visual manner with the intent of leveraging our
natural visual capabilities. Visualization allows us to process, analyse and
understands massive amounts of data very quickly. The problem with
visualizations is that we are trying to use them, to quickly understand more and
more abstract data, that does not always have an immediately obvious mapping to
the real world. This presents a challenge in trying to develop reusable
meaningful visualisations. There is ongoing research to try and leverage
metaphors and to try understand what type of analytic tasks could be enhance by
visualizations based on those metaphors.

The paper talks about visualization developments in several areas including,
document and software visualization, distortion techniques, hierarchies, data
integration (including data warehousing and data mining) and the world wide web.
At the time of publication of this paper in the late 90's there was also some
real world commercial applications including applications based on work done at
Xerox PARC including Perspective Wall, Cone Tree and Wide Widgets.

The author suggests that, at the time, there was still much work to be done in
terms of understanding the potential and usefulness of the technology but it was
clear that visualizations would play a huge role in the future of computing.
Gershon describes visualization as an emerging discipline which must progress
through four fundamental stages and suggests that visualization is progressing
at an accelerated pace because many of the step are happening in parallel; ``An
emerging discipline progresses through four stages. First, it starts as a craft,
practiced by artisans using heuristics. Second, researchers formulate scientific
principles and theories to gain insights about the processes. Third, engineers
eventually refine these principles and insights into production rules. Fourth,
the technology becomes widely available.''

The paper suggests that widespread usefulness of visualizations will be directly
influenced by the content providers' (software design and developers) ability to
create visualizations that are accessible and easily understood by a wide range
of end users, making them equally useful for users of diverse backgrounds. In
order to create these meaningful visualizations in real world applications the
systems being modeled and the processes involved must be understood. This
understanding in itself could lead to the correct visualization for any
particular task.

There are many areas that we need to learn more about in order to better support
human learning through visualizations such as what types of interactions make
sense for any particular visualization, how we perceive and understand
information both visually and non-visually as well as how we search for
information and how we adapt those searches based on previous knowledge. A
better understanding of these issues and how they relate to specific analytic
tasks will inform the creation of more useful, meaningful visualisations that
that can truly help end users process more data more accurately and quickly. The
ability to create more flexible task oriented user interfaces can also aid in
the development of more useful visualizations.

An important point that Gershon makes in the paper is that we also need to
understand when graphic visualization may, in fact, be less appropriate.
Sometimes a visualization may be harder to understand than words; we need to
weigh the options and choose the representation that is most appropriate to
allow the users to accomplish the required task.

In conclusion the paper re-iterates that there is great need for the evolution
of this emerging technology especially in ``The development of scientific and
engineering principles for the generation of visualizations  (to users with
diverse needs and capabilities) and a methodology for solving problems with
information visualization are badly needed.''

\subsubsection{Key Contributions}
\subsection{Summary: Situational Awareness Support to Enhance Teamwork in
Collaborative Environments \cite{Kulyk:2008:SituationalAwareness}} This paper
was motivated by that fact that today, many modern teams are collaborating using
a multitude of visualizations on multiple large format displays. During these
activities the situational awareness of the teams is not always ideal. This
paper attempts to develop some guidelines to help design better co-located
collaborative systems for use with multiple large format displays with
affordance for situational awareness.

There were several concepts developed to help support situational awareness on
large shared displays including an interface the authors called ``Memory Board
is an interface that automatically stores and visualizes the activity history''.
It provides participants access to historical annotations made on other views or
visualizations and allows them to be aware of which member is controlling a
particular visualisation or display. They also developed the \emph{Highlighting
on Demand} interface which allows individuals to highlight or fade out any part
of any display using a personal interaction device.

The paper argues that simply providing complex visualizations across or using
multiple displays may, in fact, increase the cognitive load and have a negative
impact on the collaborative activity. Designing situational awareness into these
applications is necessary to properly coordinate team decision making and truly
support the collaborative process. One reason is that the participants can be
easily distracted by sub-tasks that should in fact be transparent to the
activity but in reality are not, like who is in control of a shared artifact or
which display or visualisation is the one currently of interest. Shared
awareness of these types of interactions are essential and can ''leads to
informal social interactions and development of shared working cultures which
are essential aspects of group cohesion''.

Kulyk states that situated awareness is concerned with the individuals knowledge
and understanding of the events, information and environment as well as the
shared understanding of the team as a whole and their understanding of the past
and present and even its impact on future events. Kulyk's work focuses on the
following three main aspects which directly relate to and extend Endsley's three
levels of situated awareness:
\begin{itemize}
\item A person's previous knowledge and understanding which includes the
source and nature of relevant events and information.
\item Detection and comprehension of the relevant perceptual cues and
information from the environment including the comprehension of multiple
visualizations and their context.
\item Interpretation of theses visualizations and the continuous reconfiguring of
understanding and knowledge during collaboration. This support the
awareness of changes in the environment as well as the ability to
keep track of work in progress.\end{itemize} 
Kulyks is actually interested in the impact of situational awareness on team
collaboration and specifically what she refers to as \emph{shared situational
awareness}. She defines it as the level of the group or team's awareness of
individual situational awareness based the above three points. With respect to
shared situational awareness, the paper is mainly concerned with answering the
following questions:
\begin{itemize}
\item To understand the impact of situational awareness on
collaboration.
\item Understand how to support shared situational awareness in collaborative
environments.
\item Understand how to leverage the usage of shared large format monitors to 
support shared situational awareness.
\item How can new interactive systems and visualisations be designed and
evaluated based on their support for situational awareness and how might these
systems actually stimulate new and existing forms of collaboration.
\end{itemize}
The paper points outs that the level of support for situational awareness
varies depending on the collaborative activity. For example a collaborative team
that supports emergency dispatch would need a higher level of support when
compared to a team working on scientific collaboration. In the long run,
whatever the reason for the collaboration, mistakes cost and better situational
awareness support can help reduce mistakes. 
Next the paper talks specifically about situational awareness and scientific
collaboration and in particular how large shared displays could be used to 
better support situational awareness in current omics experimentation in 
molecular biology. The following key points were made:
\begin{itemize}
  \item Visualizations on a shared display encourages group discussions.
  \item The visualization of data on a shared display allows users to quickly
  determine the quality of the information or data-set.
  \item Multiple visualizations may be needed and when used there must be a
  clear visual relationship between these visualizations so as to avoid
  participants from getting lost, from being distracted and from change blindness
  which occurs when the participants to do not realize that the focus of the
  discussion has changed from one visualization to  another.
  \item When multiple shared displays are being used with multiple
  co-related visualizations then changes in one visualization should also trigger
  changes in the other related visualizations.
  \item Difficulties arise when team members from diverse disciplines need to 
  use the same visualizations that may have been designed to be understood by
  people with a specific background or skill-set as is the case in many
  scientific visualizations.
\end{itemize}
To address some of the problems related to situational awareness for shared
displays the authors implemented Highlighting on demand interface which allowed
the person currently controlling the shared display to highlight or fade out
certain visualizations or areas of the shared display(s). This would increase
awareness among the team as to the relevant visualization at that moment. A
second solution was the memory board interface which stores and visualizes the 
changes in real-time during team collaboration. This ensures that the entire
collaborative process is recorded and all intermediate visualizations,
annotations etc. can be retrieved and reviewed. A control interface is
suggested as a means of controlling the shared displays and visualizations and
giving access information about the visualizations and displays. This
was envisioned to run on a shared touch screen as well on the team members
individual interaction devices. Knowing who is making changes and what changes
are being made is critical to any collaborative effort and the control interface
attempt to address this problem.
\subsubsection{Key Contributions}
\subsection{Sumary: Beyond Models and Metaphors: Visual Formalisms in User
Interface Design\cite{Nardi:BeyondModels}}
\subsubsection{Sumary}
In this paper they argue that the user interface can be designed to support
the visual representation of the underlying program's semantics and can help
offload cognitive load to facilitate problem solving. Nardi postulates that at
the time this paper was published in the early 90's it was time to start
focusing more attention on the semantic interface rather then the also
important but less interesting supply of user commands to the underlying
application. She claims that design and analysis tasks are complex and software
designed to support these activities needs to be enhanced with well designed
visual interfaces from which the semantics of the application could be easily
inferred by the users. The designers of this type of complex application would
be well served to have a toolkit of computational building blocks, such that
they could piece together the appropriate blocks which could support this
visual transfer of application semantics.

Nardi argues that neither Mental models nor Metaphors are appropriate
theoretical foundations to support such  building blocks but that these blocks
could be developed by leveraging the power of visual formalisms. She argues that
different visual formalisms can be used as the building blocks for semantic
interfaces and showed how a table or grid visual formalism was implemented and
explains the potential for it to be used as a building block.
\subsubsection{Key Contributions}
\subsection{Summary: Heuristics for Information Visualization Evaluation}
This paper defines heuristic evaluation as a quick and lightweight method to
predict and spot issues related to the usability of an applications' user
interface. The heuristic approach involves a small  group testing the system
based on the heuristics or guidelines that are relevant to the system.
Heuristics can often be used to teach novice users and as a method of
identifying design patterns. Like design patterns heuristics help by providing
an over arching language from from which the system can be talked about and
promote reuse especially through the evolution of design patterns. Zuk describes
heuristic evaluation as commonplace in the field of HCI but, its application to 
information visualization is still emerging.  The author states that we are at a
point where there is a need to create a small set of general  overarching
heuristics that could be applied to most information visualization system, 
similar to Neilson's 10 usability heuristics for the design of user interfaces.
The paper describes this as an exciting and open area for future study. Zuk
suggests  that a hierarchical grouping of existing heuristics combined with
different tree searching algorithms could help in selecting a core set of
heuristics. Otherwise,  one could follow Neilson's method of reducing a large
set of problems into a small set of easily understandable heuristics. Next the
paper describes the process of heuristic evaluation as something that is apt to
evolve but nonetheless describes that an iterative approach using five
evaluators has been found to be effective. In the first pass the evaluators try
and get a general sense of the application and a second pass is used to focus on
applying the heuristics to  specific elements of the user interface. The exact
number of evaluators will most likely be application specific and dependant on
the complexity of the visualizations. The paper refers to work by Tory and
Moller  /cite{} who suggest using a combination of traditional usability experts
and visualization or data display experts. It is still to be determined what the
exact characteristics of  information visualization evaluator should be.  The
paper describes work done by Craft and Cairns \cite{} as determining that there
is still much work to be done in this area, "They conclude by calling for a more
rigorous design methodology that: takes into account the useful techniques that
guidelines and patterns suggest, has measurable validity, is based upon a
user-centred development framework, provides step-by-step approach, and is
useful for both novices and experts." The paper also warns that there could be
problems if we blindly apply existing knowledge and methods of usability
heuristics to information visualization and suggest that caution should be used
and further study is needed. Zuk noted in the case study that the evaluators had
problems with the heuristics that were too specific and felt a more general
vocabulary would increase its usefulness. It is was also noted that in general
more precise, clear and understandable descriptions of each heuristics would
facilitate their use and help ensure that they are used as intended and it
leaves less room for misinterpretation. The case study revealed that some
problems were detected by distinct heuristics which the author suggested could
be used as way to see the problem from distinct points of views and possibly
could lead to multiple solutions.  It was also found that assigned priorities
could help resolve issues when conflicting heuristics were found. Another
suggestion was to let domain experts resolve heuristic conflicts. The authors
also identified some problems that may be common among visualizations in
general. Some of these issue include: \begin{itemize} \item Poor contrast can
produce hidden or difficult to see visual components. \item Poor choice of
colors could lead to confusion in terms of understanding or misinterpreting the
relationships between visualizations. \item Tool tips lack the detail required
to make them useful.
\end{itemize}
The paper notes that some heuristics would be better used as guidelines for the
designers and or developers of the system and other might be better for
heuristic evaluation of the visualization. Zuk suggests that paired evaluator
teams could be comprised of evaluators paired with domain experts, this could
help in a more accurate application of the heuristics. It was also noted that
usability heuristics could could be useful in combination with the visualization
ones but note that this requires further study. Another open research are would
be to discover if tools could be used to support the evaluation of
visualizations.
\subsubsection{Key Contributions}
\subsection{Summary: A Collaborative Dimensions Framework: Understanding the Mediating Role of Conceptual Visualizations in Collaborative Knowledge Work}
The goal as stated in the paper "is to identify the factors that contribute to
the choice of an effective visual representation for collaborative knowledge
work to support distributed cognition, and to organize them in a conceptual
framework".  Bresciani  focuses on visualizations as artefacts and claims that
this focus allows for decision factors to be expressed in terms of dimension
that either support  or interfere  with certain kinds of visualization uses.
These dimensions are not independent and often, supporting one dimension will
have a negative impact on the support for another. Understanding the
relationships between dimension allows designers of collaborative visualizations
to pick and leverage those dimension that make the most sense for the
application being developed. The author suggests  that this approach will help
build better support for collaborative activities then could be achieved using
general guidelines that should apply to all visualizations as might be the case
using a heuristic approach. This paper leverages research from three main fields
including:
\begin{itemize}
\item Blackwell's work on the cognitive dimensions of notations framework
\cite{Blackwell:NotationalSystems} and Hunhausen's work on communicative dimensions framwork \cite{Hundhausen:CommunicativeDimensionsFramework}
\item Star's research on the boundary object paradigm \cite{Star:BoundaryObjects} 
\item  Thirdly a literature review on visualization technologies, in particular the information visualization work of Shneiderman and Karabeg as well as the knowledge visualization work of Eppler and Suthers.
\end{itemize}


Informed by  this theoretical foundation and a combination of techniques used to
gather the experiences and knowledge of expert practitioners who employ
visualizations in their daily collaborative activities, a collaborative
visualizations framework was developed to help understand how visualizations can
be leveraged to support collaborative activities and how they can facilitate the
distribution of knowledge across team members. The framework consisted of the
following seven dimensions:

\begin{itemize}
\item \emph{Visual impact}: The extent to which the visualization is attractive and if it facilitates attention and recall.
\item \emph{Clarity}: Is the visualization self-explanatory, easy to understand and dose it reduce cognitive effort.
\item \emph{Perceived finishedness (provisionality)}: Is the visualization perceived by the user as a final product.
\item \emph{Directed focus}:Does the visualization direct the users attention to any particular area or areas.
\item \emph{Inference support}: Do the constraints of the visualization help or  support the creation of new and novel insights. "Inference support is the core differentiator and added value of visualization over text: it allows to gain new understanding "for free" just by changing the visualization type, the focus, or the representational constrains."
\item \emph{Modifiability}: Is there support for dynamic modification of the visualization?
\item \emph{Discourse management}: Does the visualization support control over the discussion and workflow of the activity.
\end{itemize}
The paper applied the dimensions to real world visualizations to determine to what extent the different dimensions were supported by the visualizations. They identified six different but common forms of collaboration and suggested that some dimensions would be more appropriate for certain collaborative activities, The following lists each of the six collaborative activities along with the dimensions that provide the best support for those activities:
\begin{itemize}
\item idea generation (high visual impact, low clarity, low perceived finishedness, high modifiability)
\item general knowledge sharing (high visual impact, low perceived finishedness)
\item problem analysis (low perceived finishedness, high clarity, high inference support)
\item option evaluation (high clarity, high directed focus, high inference support, moderate-low modifiability)
\item deliberation or decision elaboration (high clarity, high directed focus, high inference support, moderate-low modifiability)
\item planning (high modifiability)
\end{itemize}




As noted the above dimensions are not always independent and careful
consideration must be made when deciding the level of support for each
dimension. This concept of trade off between dimensions was first introduced by
Green and Petre \cite{}. As an example of these trade-offs the paper refers to
the fireworks visualization from Let's Focus collaborative software which was
used as one of the three visualizations to which they applied the collaborative
dimensions.  The application of the dimensions to the fireworks visualization
revealed several relationships or trade-offs between the dimensions. Visual
impact had effect on clarity and on directed focus. The high visual impact of
the fireworks visualization was distracting and therefor reduced its clarity and
it had reduced support for directed focus because of the appealing nature of the
visualization in fact, "When the visual stimuli are very high, then the focus
will diminish because the attention is caught more by the aesthetics than the
content". There were also trade-offs between modifiability and perceived
finishedness and  between modifiability and discourse management. The
visualization had a high level of support for discourse management through the
Let's Focus software package that provided the visualization. It also had high
perceived finishedness "because the diagram resembles a final piece of work
instead of a discussion tool." If a visualization is perceived as finished, then
it will be less likely that collaborators will feel like they should contribute
or modify it. If there is no tool support for discourse management then high
modifiability usually results in in low discourse management because
coordination of the group becomes more difficult.


Another important consideration is the medium on which the visualization is
delivered.  "The choice of alternative media such as a whiteboard, paper or
computer-based interaction, strongly affects people's willingness to make
contributions and collaborate." If people feel that modifications of the
visualization are permanent then they are less likely to want to contribute,
also the perceived finishedness can be affected by the medium, a peice of paper
tends to be though of as more provisional where as a visualization on a computer
monitor has more of a sense or permanence even though the computer based
visualization may in fact have high support for modification.
\subsubsection{Key Contributions}
\subsection{Sumary: Distributed and Collaborative Visualization}
\subsubsection{Key Contributions}
\section{Discussion}
\section{Conclusions}
\bibliographystyle{abbrv}
\bibliography{litReview}%sigproc.bib is the name of the Bibliography
\end{document}


